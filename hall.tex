%!TEX root = ../hall.tex
% Тип документа
\documentclass[a4paper,12pt]{extarticle}

% Шрифты, кодировки, символьные таблицы, переносы
% \usepackage{cmap}
% \usepackage[T2A]{fontenc}
\usepackage[utf8]{inputenc}
\usepackage[russian]{babel}
% Это пакет -- хитрый пакет, он нужен но не нужен
\usepackage[mode=buildnew]{standalone}

\usepackage
	{
		% Дополнения Американского математического общества (AMS)
		amssymb,
		amsfonts,
		amsmath,
		amsthm,
		% Пакет для физических текстов
		physics,
		% misccorr,
		% 
		% Графики и рисунки
		wrapfig,
		graphicx,
		subcaption,
		float,
		tikz,
		tikz-3dplot,
		caption,
		csvsimple,
		color,
		booktabs,
		geometry,
		% 
		% Таблицы, списки
		makecell,
		multirow,
		indentfirst,
		%
		% Интегралы и прочие обозначения
		ulem,
		esint,
		esdiff,
		% 
		% Колонтитулы
		fancyhdr,
	}  
\usepackage{pgfplots,pgfplotstable,booktabs,colortbl}
\usepackage{xcolor}
\usepackage{hyperref}
\usepackage{pythontex}
 % Цвета для гиперссылок
\definecolor{linkcolor}{HTML}{000000} % цвет ссылок
\definecolor{urlcolor}{HTML}{799B03} % цвет гиперссылок
 
\hypersetup{pdfstartview=FitH,linkcolor=linkcolor,urlcolor=urlcolor, colorlinks=true}
\hypersetup{pageanchor=false}
% Увеличенный межстрочный интервал, французские пробелы
\linespread{1.3} 
\frenchspacing 

 
% \usetikzlibrary
% 	{
% 		decorations.pathreplacing,
% 		decorations.pathmorphing,
% 		patterns,
% 		calc,
% 		scopes,
% 		arrows,
% 		fadings,
% 		through,
% 		shapes.misc,
% 		arrows.meta,
% 		3d,
% 		quotes,
% 		angles,
% 		babel
% 	}
% Среднее <#1>
\newcommand{\mean}[1]{\langle#1\rangle}

\begin{pycode}
##
def frexp10(decimal):
	parts = ('%e' % decimal).split('e')
	return float(parts[0]), int(parts[1])
##
\end{pycode}
% Функция для тех, кто использует pythontex. Представляет любое вещественное число в стандартном виде.
\newcommand{\frexp}[1]{
		\pyc{#10=frexp10(#1)} 
			\py{ round(#10[0],2)} 
				\cdot 10^{\py{#10[1]}} }

% const прямым шрифтом
\newcommand\ct[1]{\text{\rmfamily\upshape #1}}
\newcommand*{\const}{\ct{const}}
\usepackage{array}
\usepackage{pstool}

\geometry		
	{
		left			=	2cm,
		right 			=	2cm,
		top 			=	2.5cm,
		bottom 			=	2.5cm,
		bindingoffset	=	0cm
	}

%%%%%%%%%%%%%%%%%%%%%%%%%%%%%%%%%%%%%%%%%%%%%%%%%%%%%%%%%%%%%%%%%%%%%%%%%%%%%%%
	%применим колонтитул к стилю страницы
\pagestyle{fancy} 
	%очистим "шапку" страницы
% \fancyhead{} 
	%слева сверху на четных и справа на нечетных
\fancyhead[R]{}%\labauthors 
	%справа сверху на четных и слева на нечетных
% \fancyhead[L]{Отчёт по лабораторной работе №\labnumber}
\fancyhead[L]{\labtheme} 
	%очистим "подвал" страницы
% \fancyfoot{} 
	% номер страницы в нижнем колинтуле в центре
\fancyfoot[C]{\thepage} 

%%%%%%%%%%%%%%%%%%%%%%%%%%%%%%%%%%%%%%%%%%%%%%%%%%%%%%%%%%%%%%%%%%%%%%%%%%%%%%%

\renewcommand{\contentsname}{Оглавление}
\usepackage{tocloft}
\usepackage{secdot}
\sectiondot{subsection}
\usepackage{gensymb}
\usepackage{textcomp}
\usepackage{pythontex}

\begin{document}
\def\labauthors{Карусевич А.А, Понур К.А.}
\def\labgroup{430}
\def\department{Кафедра квантовой физики}
\def\labnumber{1}
\def\labtheme{Движение носителей заряда в электрических и магнитных полях}

\renewcommand{\Re}{\operatorname{Re}}
\renewcommand{\Im}{\operatorname{Im}}
\renewcommand{\phi}{\varphi}
\renewcommand{\hat}{\widehat}

\begin{titlepage}

\begin{center}

{\small\textsc{Нижегородский государственный университет имени Н.\,И. Лобачевского}}
\vskip 1pt \hrule \vskip 3pt
{\small\textsc{Радиофизический факультет. Кафедра Радиотехники.}}

\vfill

{\Large Отчет по лабораторной работе №\labnumber\vskip 12pt\bfseries \labtheme}
	
\end{center}

\vfill
	
\begin{flushright}
	{Выполнили студенты \labgroup\ группы\\ \labauthors}%\vskip 12pt Принял:\\ Менсов С.\,Н.}
\end{flushright}
	
\vfill
	
\begin{center}
	Нижний Новгород, \the\year
\end{center}

\end{titlepage}


% \tableofcontents
% \newpage


\section{Теоретическая часть}
\subsection*{Введение}
	Особенности движения носителей заряда в электрических и магнитных
полях определяют специфику функционирования подавляющего большинства
приборов современной микроэлектроники. Данное описание содержит краткое
изложение элементарных основ теории явлений переноса носителей заряда в 
однородном полупроводниковом материале. При этом речь пойдет как о движении
в электрических полях различной напряженности, однородно и неоднородно
распределенных в пространстве, так и о движении в скрещенных электрических
и магнитных полях, т.е. в условиях проявления эффекта Холла.
Для работы любой радиолинии необходимо, чтобы ток возбуждения антенны на её передающем конце отображал передаваемый сигнал, т.е. необходимо каким-то образом «записать» его на токе высокой частоты.



\subsection{Движение носителей заряда в полупроводниках}
Описание движения электрона в полупроводнике представляет собой достаточно сложную задачу, так как кроме внешних приложенных электрических и
магнитных полей на электрон действует поля со стороны ионов, образующих
кристаллическую решетку, и оставшихся электронов. Для того чтобы последующий материал был более доступен, мы последовательно рассмотрим движение
электрона в идеальной кристаллической решетке, затем в решетке с дефектами и
закончим данный раздел описанием движения ансамбля электронов в реальных
полупроводниках.

\subsubsection{Полуклассическая модель движения электронов в полупроводниках с идеальной кристаллической решеткой}

Идеальная кристаллическая решетка представляет собой совокупность
атомов, периодически (с периодом, равным периоду кристаллической решетки
а) расположенных в пространстве. Потенциальная энергия электрона $V(\vec r)$ в
идеальной кристаллической решетке также является периодической функцией с
периодом равным периоду кристаллической решетки: $V(\vec r)=V(\vec r + \vec a)$. Теорема
Блоха утверждает, что собственные функции электрона, движущегося в таком
периодическом поле, представляют собой модулированные плоские волны вида

\begin{equation}
	\Psi_k (\vec r)= e^{i(\vec k, \vec r)} U_k(\vec r), 
\end{equation}
где $U_k(\vec r)$ периодическая функция координат с периодом прямой решетки, $\vec k$
- вектор, характеризующий квантовое состояние электрона в кристалле, имеющий размерность волнового вектора и поэтому названный квазиволновым вектором. Функции $\Psi_k(\vec r)$ называют \textit{блоховскими функциями}. Можно ввести понятие квазиимпульса электрона с помощью соотношения соотношения $\vec p = \hbar \vec k$. При заданном значении $\vec k$ имеется много решений уравнений Шредингера, и описание энергетических уровней электрона в периодическом потенциале осуществляется посредством семейства непрерывных функций $\Psi_n(\vec k)$ Совокупность всех электронных уровней, описываемых функцией $W_n(\vec k)$ при фиксированном $n$, называют \textit{разрешенной энергетической зоной} с номером $n$. В дальнейшем номер разрешенном
зоны $n$ будем опускать. Энергия электрона $W(\vec r)$ в разрешенной зоне является
периодической и четной функцией в пространстве обратной решетки \cite{lit1},\cite{lit2} , т.е.
$W(\vec k)= W(\vec k + \vec G)$  (где $G$ -- вектор обратной решетки) и 
$W(- \vec{k})=W(\vec k)$. Заметим, что всегда можно рассматривать не бесконечное множество значений вектора $\vec k$, а ограничиться изменением компонент $\vec k$  в пределах зоны Бриллюэна, включающей в себя все физически неэквивалентные значения этого вектора.

На уровне, заданном номером зоны п и квазиволновым вектором к электрон имеет отличную от нуля среднюю скорость
\begin{equation}
	\vec v (\vec k)= \frac{1}{\hbar} \nabla_{\vec k} W(\vec k)
\end{equation}

Это очень интересный результат. Согласно ему электрон в периодическом потенциале имеет стационарные уровни, находясь на которых он, несмотря на
взаимодействие с периодической последовательностью ионов, продолжает двигаться бесконечно долго, не теряя свое средней скорости.

В общем случае состояние электрона описывается с помощью волнового
пакета, состоящего и. блоховских функций. Если ширина пакета по квазиволновым векторам мала, по сравнению с зоной Бриллюэна и $W(\vec k)$ мало меняется для
уровней, входящих в волновой пакет, то скорость движения электрона есть ни
что иное, как групповая скорость движения центра волнового пакета

\begin{equation}
	\vec v_g(\vec k)=\frac{1}{\hbar} \nabla_{\vec k} W(\vec k)
\end{equation}
	Если полупроводник находится во внешнем электрическом или магнитном поле
то для описания изменения квазиволнового вектора $\vec{k}$ электрона и его координаты $\vec r$ можно воспользоваться полуклассической моделью. Она справедлива в
случае, когда внешние электрические и магнитные поля медленно меняются в
координатном пространстве на расстояниях порядка размера элементарной
ячейки. Тогда при известной зависимости энергии электрона в разрешенной зоне
$W(\vec k)$ состояние электрона описывается его квазиволновым вектором $\vec k$, а также
координатой $\vec r$. Считается, что в присутствии внешних электрических и магнитных полей $E$ и $B$

\begin{enumerate}
	\item номер зоны электрона не меняется (т.е. в модели пренебрегается 
	возможностью межзонных переходов);
	\item изменения квазиволнового вектора и координаты электрона определяю icm
уравнениями движения
\end{enumerate}

\begin{equation}
	\label{eq:1.4a}
	\dv{\vec r}{t}=\vec v_g (\vec k)=\frac{1}{\hbar} \pdv{W(\vec{k})}{\vec{k}},
\end{equation}
\begin{equation}
	\label{eq:1.4b}
	\hbar \dv{\vec{k}}{t}= -e \qty(\vec{E}+ \qty[\vec v_g, \vec B] )
\end{equation}

Несмотря на сложность зависимости энергии электрона от квазиволнового
вектора $W(\vec k)$ в большинстве задач физики полупроводников играет роль поведение электрона в достаточно узкой области значений квазиволнового вектора в
зоне Бриллюэна вблизи минимума или максимума энергии. Вблизи точки экстремума $\vec k_0$ функцию $W = W(\vec k)$ можно разложить в ряд Тейлора, использую
выражение
\begin{equation}
	W(\vec k)= W(\vec k_0)+\frac12 \sum_{i=x}^z \sum_{j=x}^z \frac{1}{m^*_{ij} }
	\hbar^2 (k_i-k_{i0})(k_j-k_{j0}),
\end{equation}
где $m_{ij}^*$ компоненты тензора \textit{эффективной массы} носителей зарядка, определяющиеся соотношением
\begin{equation}
	\frac{1}{m^*_{ij}}=\frac{1}{\hbar^2} \pdv[2]{W(\vec k)}{k_i}{k_j}\eval_{\vec{k}=\vec{k_0}}
\end{equation}
Выбрав соответствующую систему координат можно свести данный тензор к диагональному виду. В простейшем случае все компоненты тензора одинаковы, и тензор вырождается в скаляр, а закон дисперсии принимает параболическую форму
\begin{equation}
	W(\vec{k})=W(\vec k_0) +\frac{1}{2m^*} \hbar^2 (\vec k - \vec k _0)
\end{equation}
Тогда уравнение движения электрона преобразуется к виду
\begin{equation}
 	\label{eq:electron_motion}
	m^* \vec a = -e \qty ( \vec E+ \qty[\vec v_g, \vec B] ),
\end{equation}
где $\vec a$ - ускорение электрона. Из формулы \eqref{eq:electron_motion} следует, что электрон в
периодическом поле идеальной кристаллической решетки при воздействии
внешнего электрического или магнитного поля ускоряется относительно решетки так, как если бы его масса была равна эффективной массе. Заметим, что
эффективная масса электрона может принимать как положительные (около дна
разрешенной зоны), так и отрицательные (около потолка разрешенной зоны)
значений одной из самых впечатляющих особенностей зоной теории твердых тел
является использование понятия дырок. Незанятые электронами (вакантные) со­
стояния называют \textit{дырочными} состояниями или просто \textit{дырками}. Оказывается,
что ток, получаемый при заполнении электронами совокупности определенного
количества уровней в зоне, в точности совпадает с тем, который можно получить, если оставить эти уровни незаполненными и заполнить все остальные состояния в зоне частицами с положительным зарядом $+e$ (противоположным
заряду электрона). Подчеркнем, что для одной и той же зоны нельзя пользоваться­ сразу двумя способами описания. Если считать, что ток переносят электроны, то незаполненные уровни не дают в него никакого вклада; если же считать, что ток переносят дырки, то отсутствует вклад от электронов Допустимо, однако, одни зоны описывать на языке электронов, а другие -- на языке дырок в зависимости от того, какой способ описания более удобен. Физические свойства дырки вытекают из факта заполненности электронами всех остальных состояний зоны.

Они обобщены в \ref{tab:1} \cite{lit1}. Для более полного изучения свойств дырок рекомендуется обратиться к \cite{lit1} или \cite{lit2}.
\begin{tabular}{c}
	\label{tab:1}
\end{tabular}

\subsubsection{Движение электронов  в реальных полупроводниках}

Поле реальной кристаллической решетки не является строго периодическим из-за присутствия в полупроводнике дефектов Взаимодействие электрона с дополнительной силой, возникающей вследствие нарушения периодичности потенциала, приводит к рассеянию электрона, т.е. изменению электроном квазиволнового вектора и даже (в случае неупругого взаимодействия) энергии. В случае самой простой классификации дефекты кристаллической решетки можно от­нести к двум разным типам. Обычно рассматриваются так называемые \textit{статические} дефекты решетки, к которым относятся точечные дефекты, дислокации и т.д. и дефекты, \textit{перемещающиеся} по кристаллу.  К последним в первую очередь относятся тепловые колебания кристаллической решетки. Первый тип
дефектов достаточно подробно был рассмотрен ранее (см. \cite{lit3}), здесь мы остановимся на особенностях второю типа дефектов кристаллическом решетки.

\subsubsection{Колебания кристаллической решетки, фононы. Взаимодействие электронов с фононами}

Известно, что атомы кристаллической решетки совершают хаотические
колебания около положений устойчивого равновесия. Коллективное движение
частиц в форме упругой волны называют нормальным колебанием кристаллической решётки. Зависимость частоты колебаний $\omega$ от волнового числа $q$ называют
\textit{законом дисперсии} для колебаний атомов.

В однородной струне как целом сплошном твёрдом теле могут возникать упругие волны, распространяющиеся со звуковой скоростью $v_\text{зв}$. При этом частота колебаний оказывается пропорциональной волновому числу: $\omega=q\cdot v_{\text{зв}}$.

Абсолютная величина волнового числа может принимать значения от $0$ до $\infty$.

Решение уравнения колебаний для линейной цепочки атомов одного сорта массой $M$ приводит к решению, соответствующему колебаниям с частотами
\begin{equation}
	\omega=\pm \omega_{max} \sin{\frac{aq}{2}}.
\end{equation}
Здесь $\omega_{max}=2\sqrt{\frac{\beta}{M}}$, $a$--период решетки, $\beta$-- коэффициент квазиупругости, характеризующий взаимодействие между частицами при отклонении от положения равновесия. Скорость распространения упругой волны в этом случае зависит от длины волны $\lambda$, что является специфическим свойством упругих волн в среде с атомной структурой, отличающим последнюю от струны (или другого твёрдого тела) как сплошной среды. Имеем
\begin{equation}
	v=\frac{\omega}{q}=\frac{\lambda}{\pi}\sqrt{\frac{\beta}{M}}\sin{\frac{\pi a}{\lambda}}
\end{equation}
Рассмотрим далее линейную цепочку атомов, в которой чередуются два
типа частиц с массами $M_1$ и $M_2$ (рис. \ref{fig:1}). Расстояние между соседними атомами
одного сорта по-прежнему равно $a$.

Решение уравнения гармонических колебаний для такой системы будет содержать две ветви с частотами

\begin{gather*}
	\omega_1^2=\frac{\omega_0^2}{2}\qty[1+\sqrt{1+ \gamma^2 \sin^2 \frac{aq}{2}	}] \\
	\omega_2^2=\frac{\omega_0^2}{2}\qty[1+\sqrt{1- \gamma^2 \sin^2 \frac{aq}{2}	}], 
	\text{ где} \\
	\omega_0^2=2 \beta \frac{M_1+M_2}{M_1M_2}, \gamma^2=4\frac{M_1M_2}{(M_1+M_2)^2} \\
\end{gather*}
\begin{figure}[h!]
\begin{minipage}[h]{0.45\linewidth}
	\centering
	\includegraphics[width=\linewidth]{example-image-a}
	\caption{а) линейная цепочка с базисом из двух различных атомов}
	\label{fig:1.1a}
\end{minipage}
\hfill
\begin{minipage}[h]{0.45\linewidth}
	\centering
	\includegraphics[width=\linewidth]{example-image-b}
	\caption{б) линейная цепочка с базисом из двух одинаковых атомов, при котором возникают оптические колебания}
	\label{fig:1.1b}
\end{minipage}
\vfill
\begin{minipage}[h]{0.45\linewidth}
	\centering
	\includegraphics[width=\linewidth]{example-image-c}
	\caption{в) поперечные оптические колебания в линейной цепочке}
	\label{fig:1.1c}
\end{minipage}
\hfill
\begin{minipage}[h]{0.45\linewidth}
	\centering
	\includegraphics[width=\linewidth]{example-image-a}
	\caption{акустические колебания}
	\label{fig:1.1d}
\end{minipage}
\end{figure}

Колебания более высокой частоты $\omega_1$  принято называть \textit{оптическими}, а 
с $\omega_2$-акустическими. При $q\rightarrow 0$ в оптической ветви колебаний атомы 
решетки смещаются в противоположных направлениях, т.е. колеблются в противофазе, так что
центр тяжести каждой пары, составляющей ячейку, остается неподвижным
(рис. \ref{fig:1.1c}). В акустической же ветви атомы смещаются в одну сторону
(рис. \ref{fig:1.1d}). На рис. \ref{fig:2} изображены соответствующие дисперсионные кривые колебаний.
\begin{figure}[h!]
	\centering
	\includegraphics[width=0.5\linewidth]{example-image-a}
	\caption{Дисперсионные кривые тепловых колебаний кристаллической решетки}
	\label{fig:2}
\end{figure}

Заметим, что выводы теории для линейной цепочки с чередующимися атомами разных масс в определенном смысле пригодны и для линейной цепочки с одинаковыми атомами, при условии, что имеются две подрешетки (рис. \ref{fig:1.1b}). В элементарной ячейке цепочки, изображенной на \ref{fig:1.1b} содержатся два атома. Оптические колебания возникают 
в результате колебания в противофазе одной подрешётки  относительно другой. В объёмном кристалле сохраняются основные закономерности, справедливые для одномерной решетки. Поэтому оптические колебания наблюдаются, в частности, в $Si$ и $Ge$, которые сдержат один
сорт атомов, но состоят из двух подрешеток (структура типа алмаза). Однако,
для объемных кристаллов, имеющих в алиментарной ячейке только один атом,
как и для простых (однородных) линейных цепочек, существуют только акустические колебания.

Энергия каждого нормального колебания квантована. Нормальные колебания  можно рассматривать подобно линейным гармоническим осцилляторам с собственной частотой $\omega_{qj}$ и энергией
\begin{equation}
	W_{qj}=\hbar \omega_{qj}\qty(n_{qj}+\frac12),
\end{equation}
где $n_{qj}=0,1,2,\dots$-- главное квантовое число $qj-$го осциллятора, колеблющегося с частотой
$\omega_{qj}$, j - индекс ветви колебаний. 

Полная энергия теплового движения атомов складывается из энергий всех нормальных колебаний
\begin{equation}
	W=\sum_{q,j} W_{qj}=\sum_{q,j} \hbar \omega_{qj} (n_{qj}+\frac12),
\end{equation}
где волновое число $q$ имеет столько разрешенных значений, сколько в кристалле
элементарных ячеек. При описании взаимодействия носителей заряда с тепловыми колебаниями решётки принято говорить о квазичастицах - фононах носителях квантов энергии колебаний решетки. При таком подходе изменение энергии колебаний решётки на один квант рассматривается как появление (или исчезновение) одного фонона с энергией и импульсом $ p = \hbar q$ . Процесс
рассеяния электронов на тепловых колебаниях решетки теперь можно рассматривать как столкновение с фононом. При таком столкновении должны соблюдаться законы сохранения энергии и импульса 
\begin{equation}
	\label{eq:lse}
	\vec k' = \vec k \pm \vec q, ~ W'= W\pm \hbar \omega_{qj},
\end{equation}
где $\vec k$, $\vec q$ - волновые векторы электрона и фонона до столкновения; $\vec k'$ - волновой вектор электрона после столкновения; $W$ и $W'$ - соответственно, энергия электронов до и после столкновения. Процесс, соответствующий в \eqref{eq:lse} знаку плюс, интерпретируется как поглощение, а знаку минус -- испускание электроном фонона.

В некоторых отношениях фононы ведут себя не так, как обычные частицы. Во-первых, среднее число фононов зависит от температуры, а во-вторых, при взаимодействии, например, с электронами или друг с другом фононы возникают и исчезают. Поэтому их называют квазичастицами.

\subsubsection{Дрейф электрона во внешнем электрическом поле}


Рассмотрим движение электрона в реальном полупроводнике во внешнем электрическом поле. Так как в реальной кристаллической структуре присутствуют дефекты частица движется ускоренно лишь на небольшом участке пути, а затем испытывает рассеяние, теряет направленную скорость, псле чего процесс разгона начинается заново. Движение электрона между актами рассеяния по прежнему описывается уравнениями  \eqref{eq:1.4a} и \eqref{eq:1.4b}. Если время свободного пробега электрона (время между двумя процессами рассеяния) мало по сравнению со временем движения электрона от одного края зоны Бриллюэна до другого, то эффективная масса электрона мало меняется во время его свободного пробега. Тогда зависимость проекции мгновенной скорости электрона на ось, направленную против внешнего электрического поля представлена на рис. \ref{fig:3}. Линейный участок на графике соответствует ускорению электрона во внешнем поле, вертикальный -- рассеянию электрона. Благодаря наличию внешнего поля электрон обладает средней скоростью направленного движения вдоль поля, которая называется \textit{дрейфовой скоростью}.

В слабых электрических полях дрейфовая скорость пропорциональна напряженности электрического поля:
\begin{equation}
\label{eq:1.14}
	v=\mu E
\end{equation}

\begin{figure}[h!]
	\centering
	\includegraphics[width=0.5\linewidth]{example-image-a}
	\caption{Зависимость проекции мгновенной скорости носителей заряда от времени}
	\label{fig:3}
\end{figure}

Коэффициент пропорциональности между скоростью и полем $\mu$ называется подвижностью носителей заряда . Эта величина численно равна средней скорости
направленного движения частиц в электрическом поле с единичной напряженностью.

Интересно, что при увеличении электрического поля дрейфовая скорость
перестает расти по линейному закону и в больших полях или стремится к установившемуся значению или уменьшается. Типичные зависимости скорости
дрейфа носителей заряда в различных полупроводниках приведены на рис.\ref{fig:4}

\begin{figure}[h!]
	\centering
	\includegraphics[width=0.5\linewidth]{example-image-a}
	\caption{Экспериментальные зависимости дрейфовой скорости носителей зарада от напряженности электрического поля для $Ge$, $Si$, $GaAs$}
	\label{fig:4}
\end{figure}

Характер зависимости $v(E)$ определяется как структурой зоны проводимости полупроводника, так и механизмами рассеяния. В валентных материалах основной причиной ограничения дрейфовой скорости является рассеяние на оптических фононах. В отличие от почти упругого рассеяния на акустических фононах, рассеяние на оптических фононах является резко неупругим, т. е. про­
исходит существенное изменение энергии носителей заряда. Как только энергия
электрона становится выше энергии оптического фонона (см. рис. \ref{fig:2}), резко
возрастает количество актов столкновений, сопровождающихся возбуждением
фонона. Таким образом, электрон активно отдаёт энергию кристаллической
решетке, что препятствует дальнейшему росту скорости его направленного движения. В режиме насыщения скорости вся энергия, набираемая электроном в электрическом поле за время свободного пробега, отдаётся им в кристаллическую решётку посредством возбуждения фононов.

В полупроводниковых соединениях ($GaAs$, $InP$ и др.) на зависимость сред­ней скорости электронов от напряженности электрического поля существенно влияет переход электронов из $\Gamma$-долины с низкой эффективной массой носителей
заряда в верхние $L$ и $X$ долины со значительно большей массой и меньшей подвижностью (средней дрейфовой скоростью) электронов \cite{lit4}-\cite{lit5}. Как правило, такой
процесс происходит при столкновении с оптическим фононом. После столкновения направленная скорость электрона в среднем теряется. В результате этих
процессов на зависимости средней скорости электронов от напряженности электрического поля появляется экстремум, а количество электронов, находящихся в
верхних долинах, (заселенность долин) растет с увеличением напряженности по­
ля.

\subsubsection{Движение носителей зарядка в плавно изменяющихся во времени или в пространстве электрических полях}


Наиболее общей системой уравнений, описывающей перемещение во
внешнем электрическом поле ансамбля электрически заряженных частиц в случае, когда электрические поля медленно меняются во времени\footnote{по сравнению со временем релаксации энергии электронов $\tau_W$} или в пространстве\footnote{по сравнению с длиной релаксации энергии электронов $l_W=\tau_W\cdot v_T$, где $v_T$-- тепловая скорость электрона при температуре T}, является система уравнений, включающая в себя

\begin{itemize}
	\item \textbf{уравнение Пуассона} (полученное из предположения о малости скорости дрейфа носителей заряда в полупроводниковом материале по сравнению со скоростью света, когда справедливо выражение $\vec E=-\nabla \varphi $)
\begin{equation}
	\label{eq:1.15}
	\Delta \varphi =- \frac{1}{\epsilon \epsilon_0} \rho(x,y,z),
\end{equation}
 где $\vec E$ - напряженность электрического поля, $\varphi$- потенциал, $\epsilon$-диэлектрическая проницаемость материала, $\rho(x,y,z)$- объемная плотность зарядка (включает в себя подвижные заряды электронов и дырок, а также неподвижные заряженные структурные дефекты полупроводника, в том числе, ионизированные примеси);

 	\item \textbf{выражения для плотности электронного и дырочного токов}

 	\begin{gather}
 		\vec j_n = -en \vec v_n + e\cdot \nabla(D_n n) \\ 
 		\label{eq:1.16a}
 		\vec j_p = ep \vec v_p - e\cdot \nabla(D_p p), \\
 		\label{eq:1.16b}
 	\end{gather}
 	n,p - концентрации, а $\vec v_n$ $\vec v_p$ - дрейфовые скорости электронов и дырок, соответственно; $e$- абсолютная величина заряда электрона; $D_n$, $D_p$- коэффициенты диффузии носителей заряда; каждое из выражений \eqref{eq:1.16b} и \eqref{eq:1.16a} для плотности тока содержит \textit{дрейфовую} (которая определяется движением носителей в электрическом поле) и \textit{диффузионную} (которая возникает из-за теплового движения подвижных зарядов) составляющие.

 	\item \textbf{уравнения непрерывности для электронов и дырок}, которые в отсутствие генерации и рекомбинации частиц записываются в виде
 	\begin{equation}
 		\label{eq:1.17}
 		\pdv{n}{t}=\frac{1}{e}\div{\vec j_n}, ~ \pdv{p}{t}=-\frac{1}{e}\div{\vec j_p}
 	\end{equation}

\end{itemize}
 	Система уравнений \eqref{eq:1.15}-\eqref{eq:1.17} дополняется \textit{соотношением Эйнштейна}, описывающим связь между подвижность и коэффициентом диффузии
 	\begin{equation}
 		D=\frac{kT}{e}\mu
 	\end{equation}
 	и выражением для дрейфовой скорости носителей зарядка как функции электрического поля
 	\begin{equation}
 		v_n=v_n(E) \text{ и } v_p=v_p(E)
 	\end{equation}
 	Тогда выражение для полной плотности тока в полупроводнике записывается в виде
 	\begin{equation}
 		\vec j = \vec j_n +\vec j_p +\pdv{E}{t},
 	\end{equation}
 	где третье слагаемое описывает ток смещения.


 	\subsubsection{Движение носителей заряда в резко изменяющихся во времени или пространстве электрических полях}


Для описания движения носителей заряда в полупроводниковых структурах при резко изменяющихся полях используется система уравнений, которая,
помимо уравнений \eqref{eq:1.15} - \eqref{eq:1.17} включает в себя уравнение баланса импульса
\eqref{eq:1.23}, уравнение баланса энергии \eqref{eq:1.24} и выражение для плотности потока
энергии \eqref{eq:1.26}. В случае транспорта электронов полная система уравнений имеет
следующий вид:
\begin{gather}
	\label{eq:1.21}
	\Delta  \varphi = -\frac{e}{\epsilon \epsilon_0} \qty(N_d-n)  \\
	\label{eq:1.22}
	\pdv{n}{t}=\frac{1}{e} \div{\vec j_n}\\
	\label{eq:1.23}
	\dv{ \qty (m^*\vec v) }{t}= -e\vec E - \frac{m^*}{\tau_p} \vec v \\
	\label{eq:1.24}
	\pdv{(Wn)}{t}=\div{\vec j_W}+ \qty(\vec j_n, \vec E)-\frac{n\qty(W-W_0)}{\tau_W} \\
	\label{eq:1.25}
	\vec j_n = - en\vec v + e\cdot \grad{(D_n n)} \\
	\label{eq:1.26}
	\vec j_W= -nW \vec v +\grad{(D_n n W)} \\ 
	\label{eq:1.27}
	\vec j = \vec j_n + \pdv{\vec E}{t} \\
	\label{eq:1.28} 
	\vec{E}=-\grad{\varphi}
\end{gather}
Здесь $N_d$ - концентрация доноров; 
$m^*$- эффективная масса электронов;
$\vec j_W$- плотность потока энергии электронов;
$\tau_p, \tau_W$- \textit{времена релаксации} импульса и энергии носителей;
$W_0=\frac32 kT$- средняя тепловая энергия электронов.

Уравнения баланса \eqref{eq:1.26} и \eqref{eq:1.24} выражают, по сути, законы сохранения
импульса и энергии частиц. Импульс направленного движения электронов может
увеличиваться за счет разгона в электрическом поле (первое слагаемое в правой
части \eqref{eq:1.23} и уменьшаться из-за рассеяния носителей на дефектах структуры
(второе слагаемое). Энергия носителей в некотором замкнутом объеме, помимо
тех же причин, может изменяться за счет втекания или вытекания горячих (т е.,
высокоэнергетических) или холодных носителей, что отражает первое слагаемое
в правой части \eqref{eq:1.24}.
В стационарном состоянии $\dv{(m^*v)}{t}=0, ~ \pdv{W}{t}=0$ уравнения \eqref{eq:1.23} и
\eqref{eq:1.24} принимают следующий вид:
\begin{gather}
	\label{eq:1.29}
	\tau_p=\frac{m^*v_s}{eE} \\
\label{eq:1.30}
	\tau_W=\frac{W_s-W_0}{eEv_s},
\end{gather}

где индекс <<s>> означает стационарное значение. Выражения \eqref{eq:1.29}, \eqref{eq:1.30}
связывают времена релаксации со стационарными значениями скорости и энергии.
Время релаксации по импульсу, как правило, много меньше времени релаксации
по энергии, т.к. упругие столкновения не изменяют энергию, но могут существенно изменить импульс частицы. На рис. \ref{fig:5} приведены графики зависимости
времени релаксации энергии и импульса в кристаллах $Si$ и $GaAs$ от величины
средней энергии носителей заряда.
\begin{figure}[h!]
	\centering
	\includegraphics[width=\linewidth]{example-image-a}
	\caption{Зависимость времени релаксации энергии и импульса от разницы между средней 
	($W$) и тепловой энергией ($W_0$) электронов в кристаллах $Si$ и $GaAs$}
	\label{fig:5}
\end{figure}
Из \eqref{eq:1.14} и \eqref{eq:1.29} следует, что 
\begin{equation}
\label{eq:1.31}
	\mu =\frac{e \tau_p}{m^*}
\end{equation}
Даже в случае полупроводников с высокой подвижностью носителей заряда,
максимальная стационарная дрейфовая скорость частиц не превышает I-
$3\cdot 10^7\frac{\text{см}}{\text{с}}$, что, казалось бы, накладывает принципиальное ограничение на быстродействие твердотельных приборов. Однако в динамическом режиме и в коротких образцах можно получить дальнейшее увеличение дрейфовой скорости электронов. Суть этого явления состоит в следующем. Когда носители попадают в
область резкого скачка поля, скорость направленного движения начинает быстро
расти у всех частиц одновременно. Поэтому средняя скорость носителей заряда в
течение короткого периода времени может быть существенно выше ее стационарного значения (\textit{эффект всплеска скорости во времени}). Затем столкновения электронов с дефектами структуры приводят к сбросу средней скорости частиц и через некоторое время (порядка нескольких времен релаксации импульса) устанавливается стационарное значение средней по ансамблю скорости носителей для данного значения поля. Таким образом, если не <<дожидаться>>
установления стационарной скорости, а использовать нестационарное значительное увеличение дрейфовой скорости частиц на временах $\sim \tau_p$, то можно получить значительное увеличение дрейфовой скорости частиц, что существенно сказывается на параметрах полупроводниковых приборов. На рис. \ref{fig:7} приведены расчетные зависимости $v(t)$ для 
$GaInAs$\footnote{$GaInAs$ -- тройное полупроводниковое соединение, отличающиеся от $GaAs$ тем, что часть атомов галлия замещена индием}, $InP$  и $GaAs$.
\begin{figure}[h!]
	\centering
	\includegraphics[width=\linewidth]{example-image-a}
	\caption{Изменение дрейфовой скорости электронов во времени после мгновенного включения электрического поля $E=40$ кВ/см. Кривые и точки соответствуют различным методам счета \cite{lit6}}
	\label{fig:figure1}
\end{figure}



\begin{thebibliography}{}
	\bibitem{lit1} а
	\bibitem{lit2} б
	\bibitem{lit3} в
	\bibitem{lit4} г
	\bibitem{lit5} д
	\bibitem{lit6} е
	\bibitem{lit7} ё
	\bibitem{lit8} ж
	\bibitem{lit9} з
\end{thebibliography}


\end{document}