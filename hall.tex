\input{text/pre}
\usepackage{pgfplots,pgfplotstable,booktabs,colortbl}
\pgfplotsset{compat=newest}
\usepackage{physics}
\usepackage{mathtools}
\mathtoolsset{showonlyrefs=true}
\newcommand\Smat{\hat { \mathbf { S } }}

\newcommand*\dotvec[1][1,1]{\crossproducttemp#1\relax}
\def\crossproducttemp#1,#2\relax{{\qty[\vec{#1}\times\vec{#2}\,]}}

\newcommand*\prodvec[1][1,1]{\crossproducttempa#1\relax}
\def\crossproducttempa#1,#2\relax{{\qty[{#1}\times{#2}\,]}}
\renewcommand{\vec}{\mathbf} % for parts

\def\Rdim{\,\frac{\text{м}^3}{\text{А} \cdot \text{с}}}
\renewcommand{\phi}{\varphi}
\renewcommand{\labelenumii}{\theenumii)} 
% \newcommand{\mean}[1]{\langle#1\rangle}
% 
% \mathtoolsset{showonlyrefs=true}
% \newcommand\Smat{\hat { \mathbf { S } }}
\let\tempint\int
\renewcommand{\int}{\tempint\limits}
\DeclareMathOperator{\Div}{div}
\DeclareMathOperator{\Rot}{rot}
\DeclareMathOperator{\Grad}{grad}
\renewcommand{\phi}{\varphi}

\begin{document}

\def\labauthors{Виноградов И.Д., Шиков А.П.}
\def\labgroup{430}
\def\labnumber{2}
\def\labtheme{Движение носителей заряда в электрических и магнитных полях. Эффект Холла}
\input{text/titlepage}

\newpage

\section*{Введение}

\section{Теоретическая часть}
\subsection*{Введение}
	Особенности движения носителей заряда в электрических и магнитных
полях определяют специфику функционирования подавляющего большинства
приборов современной микроэлектроники. Данное описание содержит краткое
изложение элементарных основ теории явлений переноса носителей заряда в 
однородном полупроводниковом материале. При этом речь пойдет как о движении
в электрических полях различной напряженности, однородно и неоднородно
распределенных в пространстве, так и о движении в скрещенных электрических
и магнитных полях, т.е. в условиях проявления эффекта Холла.
Для работы любой радиолинии необходимо, чтобы ток возбуждения антенны на её передающем конце отображал передаваемый сигнал, т.е. необходимо каким-то образом «записать» его на токе высокой частоты.



\subsection{Элементарная теория эффекта Холла}
Анализ транспорта носителей в полупроводниковых структурах, представленный в предыдущем разделе, требует знания концентрации носителей заряда и их подвижности в материале. Эти характеристики являются важными физическими величинами, определяющими многие свойства полупроводников, например, электропроводность, теплопроводность, термо-ЭДС и др.

Концентрацию и подвижность в отдельности можно определить, зная соотношение между ними. В данной работе это соотношение устанавливается экспериментально при помощи эффекта Холла.

Эффект Холла представляет собой поперечный гальваномагнитный эффект, суть которого заключается в следующем: если поместить полупроводниковую пластину во внешнее магнитное поле $\vec B$ (рис. \ref{fig:8}) и пропустить вдоль нее ток, то вследствие смещения движущихся зарядов к одной из граней пластины возникает поперечная разность потенциалов, называемая \textit{ЭДС Холла}. При этом (см. рис. \ref{fig:8}.б, \ref{fig:8}.в), носители различных знаков смещаются к одной и той же боковой грани полупроводника, поэтому с изменением типа электропроводности меняется и знак ЭДС.

С помощью эффекта Холла можно экспериментально определить тип носителей, концентрацию и подвижность в данном полупроводниковом образце. Другим важным практическим приложением этого эффекта являются измерения силы тока и мощности в цепях постоянного и переменного тока  (вплоть до очень высоких частот), напряженности постоянных и переменных магнитных полей, преобразование сигналов, анализ спектров и т.д.

\begin{figure}[h!]
\begin{minipage}[h]{0.329\linewidth}
		\centering
	\includegraphics[width=\linewidth]{fig/21a}
\end{minipage}
\begin{minipage}[h]{0.329\linewidth}
		\centering
	\includegraphics[width=\linewidth]{fig/21b}
\end{minipage}
\begin{minipage}[h]{0.329\linewidth}
		\centering
	\includegraphics[width=\linewidth]{fig/21c}
\end{minipage}
	\caption{Возникновение ЭДС Холла: схема эксперимента (а); смещение носителей заряда в дырочном (б) и электронном (в) полупроводниках, соответственно}
	\label{fig:8}
\end{figure}

Разберем эффект Холла более подробно. На рис. \ref{fig:8}.а показан полупроводник, две плоскости которого подключены через омические (т.е. невыпрямляющие) контакты к внешней батарее. Обозначим $\vec j$ плотность тока в направлении Ox. Магнитное поле $\vec B$ приложено в направлении Oy. Рассмотрим электрон, двигающийся в отрицательном направлении оси Ox со средней скоростью $\vec V$. На движущийся в магнитном поле электрон действует магнитная составляющая силы Лоренца:
$$\vec F = -e [\vec v, \vec B].$$
В результате действия этой силы траектория электрона будет искривляться  в направлении оси z, и, поскольку в этом направлении ток протекать не может, электроны будут накапливаться на боковой поверхности ($z=\pm a$, см. рис. \ref{fig:8}) до тех пор, пока не установится электрическое поле $\vec E_H$, достаточное для создания силы. равной магнитной составляющей силы Лоренца, но направленной противоположно. Приравнивая эти силы, получим: 
\begin{equation}
\label{eq:2.1}
	\vec E_H=[\vec v, \vec B]
\end{equation}

Воспользуемся законом Ома в дифференциальной форме:
\begin{equation}
\label{eq:2.2}
	\vec j = \sigma \vec E,
\end{equation}
где $\sigma = e \cdot n \cdot \mu_n$ - удельная проводимость образца, $\mu_n = \frac{v}{E}$ - подвижность носителей. Соотношение \eqref{eq:2.2} перепишем в следующем виде:
\begin{equation}
\label{eq:2.3}
	\vec j = e \cdot n \cdot \mu_n \cdot \vec E = -e \cdot n \cdot \vec v
\end{equation}

Исключая $v$ из соотношения \eqref{eq:2.1}, получим:
\begin{equation}
\label{eq:2.4}
	\vec E_H = -\frac{1}{en} [\vec j, \vec B]=R[\vec j, \vec B]
\end{equation}

Учитывая, что полный ток через образец $I=jab$, а поперечная ЭДС $U_H=E_Ha$, получим соотношение, связывающее ЭДС Холла с величиной электрического тока:
\begin{equation}
\label{eq:2.5}
	U_H=R \cdot \frac{I\cdot B}{b}
\end{equation}

Величина R называется \textit{постоянной Холла} и определяется как

\begin{equation}
\label{eq:2.6}
	R=-\frac{1}{e\cdot n}
\end{equation}

Поперечную ЭДС $U_H$, ток I, напряженность магнитного поля B (для немагнитных образцов) и толщину b полупроводникового образца можно измерить. Это позволяет найти численное значение постоянной Холла.

В действительности, произведенный элементарный вывод коэффициента Холла \eqref{eq:2.6} неточен: в нем не учтена разница между мгновенной скоростью электронов, входящей в выражение магнитной составляющей силы Лоренца, и дрейфовой скоростью, которую электрон приобретает под действием электрического поля. Кроме того, не учитывается распределение электронов по скоростям и механизмы рассеяния носителей. Формула \eqref{eq:2.6} оказывается справедливой только для металлов и вырожденных полупроводников (вырожденным называется полупроводник с очень высокой, порядка $10^{19}$ атом/$\text{см}^3$, концентрацией примеси). Более строгий анализ дает для невырожденных полупроводников значение R, которое отличается от выражения \eqref{eq:2.6} множителем А. Если учитывать рассеяние носителей только на кристаллической решетке (взаимодействие с фононами), то $A=\frac{3\pi}{8}$. В общем виде постоянная Холла может быть записана как:
\begin{gather}
	R=-\frac{A}{n\cdot e} \text{(для полупроводника n-типа)} \notag \\
	R=\frac{A}{p\cdot e} \text{(для полупроводника p-типа)}
\label{eq:2.7}
\end{gather}
где множитель А может принимать значения от 1 до 1.7. Знак минус в формуле \eqref{eq:2.7} демонстрирует, что ЭДС Холла для электронного полупроводника имеет полярность, противоположную полярности для дырочного полупроводника.

Знание электропроводности и постоянной Холла позволяет найти как концентрацию носителей, так и их подвижность.

Обозначим через холловский угол $\theta_H$ малый угол, который образует с осью х вектор напряженности суммарного электрического поля (см. рис. \ref{fig:8}):
\begin{equation}
\label{eq:2.8}
	\theta_H \cong \tg{\theta_H}=\frac{E_H}{E}
\end{equation}

Из \ref{eq:2.8} с учетом \ref{eq:2.2} и \ref{eq:2.4} получим:
\begin{equation}
\label{eq:2.9}
	\theta_H = \mu_{nH} \cdot B
\end{equation}
где $\theta_H$-холловский угол в проводнике n-типа, а $\mu_{nH}$ - так называемая \textit{холловская подвижность} электронов (индекс H указывает на метод определения подвижности). Численное значение холловской подвижности может расходиться с величиной подвижности, определенной другими методами (например, прямым способом, основанным на измерении времени распространения носителей тока по полупроводнику на определенное расстояние с известным ускоряющим полем). Последняя называется дрейфовой подвижностью. Дрейфовую подвижность можно определить из выражения \ref{eq:2.4}, если, используя выражение \ref{eq:2.7}, преобразовать его к виду:
\begin{equation}
\label{eq:2.10}
	\vec E_H = -\frac{A}{en}\cdot[\vec j, \vec B]=-A\cdot \mu_{nd} \cdot [\vec E,\vec B],
\end{equation}
где индекс d при $\mu_{nd}$ указывает, что это дрейфовая подвижность электронов.

Из выражений \eqref{eq:2.8}-\eqref{eq:2.10} следует, что для электронов $\mu_{nH}=A\cdot \mu_{nd}$, а для дырок $\mu_{pH}=A\cdot \mu_{pd}$. Используя выражения \eqref{eq:2.2} и \eqref{eq:2.7}, получим:
\begin{equation}
\label{eq:2.11}
	\mu_{(n,p)H}=R\cdot \sigma.
\end{equation}

Приведенные выше выражения относились к полупроводникам, у которых концентрация неосновных носителей пренебрежимо мала по сравнению с концентрацией основных (униполярная проводимость). Расчет постоянной Холла для материала со смешанной проводимостью приводит к формуле:
\begin{equation}
\label{eq:2.12}
	R= \frac{A}{e}\cdot \frac{n\mu^2_{nd}-p\mu^2_{pd}}{(n\mu_{nd}+p\mu_{pd})^2}.
\end{equation}
для собственного полупроводника $(n=p=n_i)$ получим:
\begin{equation}
\label{eq:2.12}
	R= \frac{A}{e}\cdot \frac{\mu^2_{nd}-\mu^2_{pd}}{\mu_{nd}+\mu_{pd})}\cdot \frac{1}{n_i}.
\end{equation}

\newpage
\section*{Эксперимент}
\textbf{Оборудование}
\begin{enumerate}
\item Источник питания образца GPS-3030D.
\item Мультиметр APPA201N в режиме измерения постоянного напряжения на пределе 200 мВ.
\item Согласующий модуль
\item Исследуемый образец ($l = 22 \text{ мм},w = 1.9 \text{ мм}, d = 0.33 \text{ мм}$)
\item Электромагнит в виде катушки с обмоткой из медного провода
\item Источник питания электромагнита GPS-3030D.
\end{enumerate}

\subsection{Схема лабораторной установки}

\begin{figure}[h!]
	\centering
	\includegraphics[scale=1.5]{ris/chem.pdf}
	\caption{Принципиальная схема включения (использовалось только направление $1\to2$)}
	\label{fig:figure2}
\end{figure}

Напряжение с источника питания GPS-3030D, работающего в режиме стабилизации напряжения, подаётся на образец через
ограничительный резистор R1. Измерение тока образца производится стрелочным миллиамперметром, находящимся на передней
панели согласующего модуля. Переключение пределов измерения миллиамперметра (10 мА – 3 мА) позволяет увеличить точность
измерения тока образца. Сопротивление миллиамперметра при работе на пределе «3 мА» – 171 Ом, на пределе «10 мА» – 51 Ом.
Для изменения направления тока через образец служит переключатель «Направление тока», имеющий среднее положение, в
котором образец отключён от источника питания.
Для измерения ЭДС Холла используется мультиметр в режиме измерения постоянного напряжения на пределе 200 мВ. 

Один из выводов мультиметра подсоединяется к контакту 3 образца, другой – к резистору R2 <<Балансировка>>.


\section{Практическая часть}
%Внимание!
%Большинство расчетов в данной лабе выполнены pythontex'ом, потому если копируете куски кода, то будьте внимательны к инородному синтаксису :) 
\subsection{Измерение ВАХ образца и паразитного напряжения на контактах}

\begin{figure}[h!]
	\centering
	\includegraphics[width = .9\linewidth]{graphs/voltamp.png}
	\caption{ВАХ образца}
	\label{fig:5.2}
\end{figure}

На рис. \ref{fig:5.2} изображена ВАХ образца (Погрешность тока $\Delta I_{sample} = I_{\max}\cdot 2.5\% = \pm 0.25
\text{ мА}$, погрешность напряжения $\Delta U = \pm 0.1 V$). Учитывая, что снимаемое напряжение - это напряжение на всей цепи, включая
образец, то напряжение на образце было рассчитано из закона Ома:
\begin{equation}
		U_{gen} = U_{sample} + I \cdot (R_1+51)
\end{equation}
\begin{equation}
	U_{sample} = U_{gen} - I \cdot (1621)
\end{equation}
По переменным напряжения и тока построена линейная регрессия,
показывающая, что данные хорошо апроксимируются прямой.


Исходя из полученной линейной зависимости (наклон прямой), учитывая погрешность измерительных приборов, найдено сопротивление образца , и
полное сопротивление цепи. Согласно схеме установки, в цепь последовательно включено сопротивление $R_1=1570$ Ом +
сопротивление амперметра, отсюда
\begin{equation}
	R_{sample}=1010 \pm25 \text{ Ом} \quad
	R_\text{цепи}=2630 \pm25 \text{ Ом}, \quad
\end{equation}
Исходя из ранее известных размеров образца: длины $l=2.2\cdot 10^{-2}$ м, ширины  $w=1.9\cdot 10^{-3}$ м и толщины $d=3.3\cdot 10^{-4} $ м, получены удельные сопротивление и проводимость материала образца:
\begin{equation}
	\rho=\frac{R_{sample}\cdot S}{l}=\frac{R_{sample}\cdot w \cdot d}{l}=
	0.029\pm 7 \cdot 10^{-4} \text{ Ом$\cdot$м}
\end{equation}

\begin{equation}
	\sigma=\frac{1}{\rho}=34.1\pm0.1 \text{ Ом$^{-1}\cdot$м$^{-1}$}
\end{equation}

\begin{figure}[h!]
	\centering
		\includegraphics[width = .8\linewidth]{graphs/paraz.png}
		\caption{Паразитное напряжение}
		\label{fig:5.3}
\end{figure}
	
На рис. \ref{fig:5.3} изображена зависимость паразитного напряжения на образце от тока, протекающего через образец.
Кривая паразитного напряжения на контактах снималась при токе в диапазоне $0\ldots10$ мА.(Погрешность тока $\Delta
I_{sample} = I_{\max}\cdot 2.5\% = \pm 0.25 \text{ мА}$, погрешность напряжения $\Delta U = \pm 0.1 mV$)
В последующих экспериментах из снятого напряжения на контактах везде вычитались значения паразитного напряжения.

\subsection{Определение типа основных носителей в образце}

\begin{figure}[h!]
	\centering
	\includegraphics[width=\linewidth]{fig/effect.pdf}
	\caption{Эффект Холла при дырочной и электронной проводимости}
	\label{fig:hall}
\end{figure}

Зная направление магнитного поля и схему включения образца, можно найти тип носителей полупроводника.

Ток течет от контакта 1 к контакту 2, милливольтметр подключен клеммой <<+>> к нижней грани образца (через балансировочную цепь к контактам 4 и 5) и клеммой
<<->> к верхней грани (контакт 3). 

Сила, действующая на заряд в магнитном поле, вызывает разделение зарядов по боковым граням полупроводника, при этом на
гранях возникает разность потенциалов (для дырочного и электронного случая на рис. \ref{fig:hall} показано разделение
зарядов)

Милливольтметр снимал положительное напряжение при данных условиях, и согласно приведенным выше соображениям, носители заряда -- дырки.

\subsection{Расчёт постоянной Холла и подвижности основных носителей}
Согласно формуле \eqref{eq:2.5}, в линейном приближении можно, зафиксировав одну из переменных (поле магнита или ток), и
снимая зависимость от другой переменной, найти постоянную Холла. Сначала фиксировался ток образца, а потом ток через электромагнит.
\subsubsection*{Фиксированный ток в образце}
 % зная величину тока или магнитного поля , можно найти  из рисунков \ref{fig:5.5} и \ref{fig:5.6} отношение постоянной Холла к его поперечному размеру.
\begin{figure}[H]
	\centering
	\includegraphics[width = .98\linewidth]{graphs/cc.pdf}
	\caption{Зависимость ЭДС Холла от напряженности магнитного поля при фиксированном значении тока образца(с учётом паразитного напряжения на контактах).}
	\label{fig:5.5}
\end{figure}

В эксперименте фиксировались значения тока образца (от 1 до 7 мА) и при них снимали зависимость напряжения
$V_{Hall}$ от напряженности поля электромагнита. Полученные зависимости построены на графике \ref{fig:5.5}( Погрешности:
$\Delta V_{Hall} = \pm 0.1 \text{ мВ}$, $\Delta B = \pm 808\cdot 0.1\text{ Гс}$, $\Delta I_\text{обр} = I_{\max}\cdot 2.5\% = \pm 0.25  \text{ мА}$ ). 

Для каждого значения тока образца была проведена линейная регрессия, исходя из неё и инструментальных
погрешностей были найдены соответствующие значения постоянной Холла:
\begin{gather}
	R_1= (8.52\pm0.4)\cdot10^{-4} \Rdim, 	\quad
R_2= (9.1\pm0.5)\cdot10^{-4} \Rdim,\\
R_{3}= (9.04\pm0.5)\cdot10^{-4} \Rdim, 	\quad
R_7= (8.44\pm0.4)\cdot10^{-4} \Rdim
\end{gather}
\begin{equation}
	\mean{R_{H}} = (8.78\pm0.3)\cdot10^{-4} \Rdim
\end{equation}

\subsubsection*{Фиксированное поле в образце}
В эксперименте фиксировались значения тока катушки (от 0.25 до 1 А, от 200 до 800 Гс соответственно) и при них снимали
зависимость напряжения $V_{Hall}$ от тока образца.
\begin{figure}[H]
	\centering
	\includegraphics[width = .96\linewidth]{graphs/cf.pdf}
	\caption{Зависимость ЭДС Холла от тока образца при нескольких фиксированных значениях магнитного поля. График построен с учётом паразитного напряжения на контактах.}
	\label{fig:5.6}
\end{figure}

Расчет постоянной Холла аналогичен предыдущему пункту. Отличие в том, что при повышении тока через образец
появляются нелинейные эффекты, и элементарная терия эффекта Холла перестает работать. Поэтому подбор линейной регрессии
осуществлялся таким образом, чтобы прямая наилучшим образом аппроксимировала экспериментальные точки на линейном участке
(см. рис. \ref{fig:5.6}). В результате расчетов получили значения постоянной Холла для каждого значения напряженности:

\begin{gather}
	R_{200}= (9.8\pm0.3)\cdot10^{-4} \Rdim, 	\quad
R_{400}= (10.7\pm0.4)\cdot10^{-4} \Rdim,\\
R_{600}= (10.3\pm0.4)\cdot10^{-4} \Rdim, 	\quad
R_{800}= (9.88\pm0.4)\cdot10^{-4} \Rdim
\end{gather}
\begin{equation}
	\mean{R_{H}} = (10.17\pm0.4)\cdot10^{-4} \Rdim
\end{equation}
\subsection{Обработка результатов}
Из полученных в предыдущих экспериментах значений можем найти среднее значение постоянной Холла:
\begin{equation}
	\mean{R_{H}\hspace{0.1em}}=(9.45\pm0.4)\cdot10^{-4} \Rdim
\end{equation}
Посчитав значение постоянной Холла и удельной проводимости, можно оценить подвижность основных носителей в образце:
\begin{equation}
	\mean{\mu_{\hspace{-0.1em}H}}= \mean{R_{H}\hspace{0.1em}}\cdot \sigma=(3.35\pm0.2)\cdot10^{-2} \,\,\frac{\text{м}^2}{\text{В}\cdot \text{c}}
\end{equation}
%% ВНИМАНИЕ!!! На самом деле в этом опыте значение R получилось в 100 раз больше. Стоит разобраться почему так
Также можно оценить концентрацию основных носителей:
\begin{equation}
	\mean{p} = \frac{3\pi}{8 R_{H}e} \approx 10^{22}  \text{ м$^{-3}$}	
\end{equation}
% section  (end)



\section{Результаты}
В данной работе был изучен эффект Холла, определен тип носителей заряда исходного образца. 
Определена постоянная Холла, оценена концентрация носителей в образце, подвижность носителей и проводимость образца
\begin{itemize}
	\item $\mean{R\hspace{0.1em}}=(9.45\pm0.4)\cdot10^{-4} \Rdim$
	\item $p\sim 10^{22} \text{ м$^{-3}$}$
	\item $\mean{\mu_{\hspace{-0.1em}H}}= \mean{R\hspace{0.1em}}\cdot \sigma=(3.4\pm0.2)\cdot10^{-2} \,\,\frac{\text{м}^2}{\text{В}\cdot \text{c}}$
	\item $\sigma=34.1\pm0.1 \text{ Ом$^{-1}\cdot$м$^{-1}$}$
\end{itemize}


\end{document}